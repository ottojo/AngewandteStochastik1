\chapter{Wahrscheinlichkeiten}
\section{Wahrscheinlichkeitsräume}
Ziel dieses Abschnittes ist die Mathematische Modellierung von Zufallsexperimenten

\subsection{Ergebnisraum $\Omega$}
Der Ergebnisraum $\Omega$ ist die Menge aller möglichen Ergebnisse $\omega$ des
Zufallsexperimentes.

\paragraph{Beispiel} Eine Münze wird geworfen. $\Omega = \{K, Z\}$ ($K$opf, $Z$ahl), oder auch
$\Omega = \{0, 1\}$.

\paragraph{Beispiel} Ein Würfel wird geworfen. $\Omega = \{1,2,3,4,5,6\}$

\paragraph{Beispiel} Anzahl täglicher Bestellungen eines Artikels:
$\Omega = \{0,1,2,\dots\} = \mathbb{N}_0$

\paragraph{Beispiel} Temperatur am Schwörmontag an der Ulmer Adenauerbrücke:
$\Omega = [-50,50]$ oder $\Omega = \mathbb{R}$
\\

Beachte:
\begin{itemize}
    \item Bei der Wahl des Ergebnisraums gibt es kein \enquote{richtig} oder \enquote{falsch}.
          Das Ziel ist, einen einfachen aber adäquaten Raum zu wählen.
      \item Der Ergebnisraum kann unterschiedliche Kardinalität haben
            (in den Beispielen: endlich, abzählbar unendlich, überabzählbar unendlich).
\end{itemize}

\subsection{Ereignisse}
Motivation: Oft ist nicht das tatsächliche Ergebnis des Experiments interessant, sondern nur
ob das Ergebnis in eine vorgegebene Menge von Ergebnissen $A$ fällt.

\paragraph{Beispiel: Würfeln} Augenzahl ist gerade: $A = \{2,4,6\}$

\paragraph{Beispiel: Anzahl täglicher Bestellungen} Vorrat von 100 Artikeln wird nicht
überschritten: $A = \{0,1,2,\dots,100\}$.

\paragraph{Beispiel: Temperatur} Temperatur beträgt über $25\degree$:
$A = [25,50]$ (falls $\Omega = [-50,50])$ oder $A= [25, \infty)$ (falls $\Omega = \mathbb{R}$)
\\

Diese Teilmengen$A$ von $\Omega$, denen Wahrscheinlichkeiten zugeordnet werden sollen,
heißen \emph{Ereignisse}.
Man sagt \enquote{Das Ergebnis A tritt ein}, falls das Ergebbnis $\omega$ des Zufallexperimentes
in dieser Menge $A$ liegt (d.h. $\omega \in A$).
Mittels Mengenoperationen können Ereignisse zu neuen Ereignissen verknüpft werden.

\begin{itemize}
    \item $A \cup B$: Ereignis $A$ oder Ereignis $B$ tritt ein.

    \item $A \cap B$: Ereignis $A$ und Ereignis $B$ treten ein.

    \item $A\setminus B$: Ereignis $A$, nicht aber Ereignis $B$ tritt ein.

    \item $A^\complement = \Omega\setminus A$: Ereignis $A$ tritt nicht ein.

    \item $\bigcup\limits_{i=1}^\infty A_i$: Mindestens eines der Ereignisse
        $A_1, A_2, \dots$ tritt ein.

    \item $\bigcap\limits_{i=1}^\infty A_i$: Alle Ereignisse $A_1, A_2, \dots$
        treten ein.
\end{itemize}

\paragraph{Beispiel} Münze wird zwei Mal geworfen.\\
$\Omega = \{KK, KZ, ZK, ZZ\}$\\
Sei $A_1$ das Ereignis \enquote{Im ersten Wurf fällt Kopf} und\\
sei $A_2$ das Ereignis \enquote{Im zweiten Wurf fällt Kopf}. Dann gilt\\
$A_1 = \{KK, KZ\}$, $A_2 = \{KK, ZK\}$.\\
Die Menge $A_1 \cup A_2 = \{KK, KZ, ZK\}$ ist das Ereignis \enquote{Es fällt mindestens ein Mal
Kopf}.\\
Die Menge $A_1 \cap A_2 = \{KK\}$ ist das Ereignis \enquote{Es fällt zwei Mal Kopf}.\\


\subsection{$\sigma$-Algebra}
Welchen Teilmengen $A$ von $\Omega$ sollen Wahrscheinlichkeiten $P(A)$ zugeordnet werden?
Ideal wäre: man definiert $P(A)$ für alle $A \subset \Omega$, das heißt für alle
$A \in \mathbb{P}(\Omega) := \{B:B\subset\Omega\}$, die Potenzmenge von $\Omega$.
Das geht, falls $\Omega$ endlich oder anzählbar ist.
Es ist aber im Allgemeinen nicht möglich
(z.B. falls $\Omega = \mathbb{R}$).
Deswegen beschränkt man sich auf ein Teilsystem $\Sigma \subseteq \mathbb{P}(\Omega)$ von
Ereignissen.
\paragraph{Forderungen an $\Sigma$:}
Mengenoperationen mit Ereignissen liefern wieder Ereignisse.

\begin{definition}
    Sei $\Omega \neq \emptyset$ eine beliebige Menge.
    Eine Menge $\Sigma \subseteq \mathbb{P}(\Omega)$ heißt eine $\sigma$-Algebra, falls
    \begin{enumerate}
        \item $\Omega \in \Sigma$
        \item Falls $A \in \Sigma$, dann gilt $A^\complement \in \Sigma$
        \item Falls $A_1, A_2, \dots \in \Sigma$, dann gilt:
            $\bigcup\limits_{i=1}^\infty A_i \in \Sigma$
    \end{enumerate}
\end{definition}

Die Elemente der $\sigma$-Algebra heißen \emph{Ereignis}.
Es folgt, dass $\emptyset$ ein Element jeder $\sigma$-Algebra ist. Dieses Ereignis heißt
\emph{unmögliches Ereignis} und tritt nie ein.
Es folgt, dass \emph{endliche Vereinigungen von Ereignissen} auch Ereignisse sind.

\paragraph{Beispiel: Münzwurf}
$\Omega = \{K, Z\}$. Die Mengen\\
$\Sigma_1=\{\Omega, \emptyset\}$ und\\
$\Sigma_2=\{\Omega, \emptyset, \{K\}, \{Z\}\} = \mathbb{P}(\Omega)$ sind $\sigma$-Algebren.\\
Die $\sigma$-Algebra $\Sigma_1$ heißt die \emph{triviale $\sigma$-Algebra}.\\
Beachte: $K \notin \Sigma_2$, aber $\{K\} \in \Sigma_2$, Ergebnisse sind \emph{keine} Ereignisse!

\paragraph{Beispiel:} $\Omega = \{1,2,3\}$\\
Dann sind die Mengen\\
$\Sigma_1 = \{\Omega, \emptyset\}$,\\
$\Sigma_2 = \{\Omega, \emptyset, \{1\}, \{2,3\}\}$,\\
$\Sigma_3 = \{\Omega, \emptyset, \{2\}, \{1,3\}\}$,\\
$\Sigma_4 = \{\Omega, \emptyset, \{3\}, \{1,2\}\}$,\\
$\Sigma_5 = \{\Omega, \emptyset, \{1\}, \{2\}, \{2,3\}, \{1,3\}, \{1,2\}, \{3\} \} = \mathbb{P}(\Omega)$
$\sigma$-Algebren

\begin{theorem}
    Für jede $\sigma$-Algebra $\Sigma$ gilt
    \begin{enumerate}
        \item $\emptyset \in \Sigma$
        \item falls $A, B \in \Sigma$ dann gilt: $A \cup B, A \cap B, A \setminus B \in \Sigma$
        \item falls $A_1,A_2,\dots \in \Sigma$, dann gilt $\bigcap\limits_{i=1}^\infty A_i \in \Sigma$
    \end{enumerate}
\end{theorem}

\subsection{Wahrscheinlichkeiten}

Wir definieren $P(A)$ für alle $A \in \Sigma$.
\begin{definition}[Disjunkte Mengen]
    Mengen $A, B$ heißen disjunkt, falls $A\cap B = \emptyset$\\
    Mengen $A_1, A_2, \dots$ heißen paarweise disjunkt (p.d.) falls für alle $i \neq j$ gilt
    $A_i \cap A_j = \emptyset$
\end{definition}

Idee: P soll folgende Eigenschaften haben:
\begin{enumerate}
    \item Für alle $A\in \Sigma$ gilt: $0 \leq P(A)\leq 1$
    \item $P(\Omega) = 1$ und $P(\emptyset) = 0$
    \item $P(A^\complement) = 1-P(A)$ für alle $A\in \Sigma$
    \item falls $A,B\in\Sigma$ disjunkt sind, so gilt
        $P(A\cup B) = P(A) + P(B)$.
    \item falls $A_1, A_2,\dots \in \Sigma$ p.d., so gilt
        $P(\bigcup\limits_{i=1}^\infty A_i) = \sum_{i=1}^\infty P(A_i)$
\end{enumerate}

Das motiviert folgende Definition

\begin{definition}[Wahrscheinlichkeitsmaß]
    Sei $\Omega \neq \emptyset$ und \tS\ eine \ts-Algebra auf \tO.
    Eine Abbildung $P: \Sigma \to [0,1]$ heißt Wahrscheinlichkeitsmaß (W-Maß) auf \tS\ falls
    \begin{enumerate}
        \item $P(\Omega) = 1$
        \item Sind $A_1, A_2,\dots \in \Sigma$ paarweise disjunkt, dann gilt
            $P\left(\bigcup_{i=1}^\infty A_i\right) = \sum_{i=1}^\infty P(A_i)$
            (\ts-Additivität)
    \end{enumerate}
\end{definition}
\noindent
Das Tripel $\left( \Omega, \Sigma, P\right)$ heißt \emph{Wahrscheinlichkeitsraum} (W-Raum).

\subsection{Beispiele zum Wahrscheinlichkeitsraum}
\subsubsection{Münzwurf}
Modellierung durch den Wahrscheinlichkeitsraum $(\Omega, \Sigma, P)$ mit\\
$\Omega = \{K,Z\},$\\
$\Sigma = \mathbb{P}(\Omega) = \{ \Omega, \emptyset, \{K\}, \{Z\}\}$\\
$P: \Sigma \to [0,1]$ mit\\
$P(\Omega) = 1, P(\emptyset) = 0, P(\{K\}) = 0.5, P(\{Z\}) = 0.5$

\subsubsection{Würfeln}
Modellierung durch den Wahrscheinlichkeitsraum $(\Omega, \Sigma, P)$ mit\\
$\Omega = \{1,2,\dots,6\}$\\
$\Sigma =\mathbb{P}(\Omega)$\\
$P: \Sigma \to [0,1]$ mit\\
$P(\Omega) = 1, P(\emptyset) = 0$,\\
$P(\{1\})=\frac{1}{6}, P(\{2\})=\frac{1}{6}, \dots$\\
$P(\{1, 2\})=\frac{1}{3}, P(\{1,2,3\})=\frac{1}{2}$\\
Also kann man kürzer schreiben $P(A) = \frac{|A|}{|\Omega|}$, mit $|A|$ der Anzahl der Elemente in $A$
(Kardinalität von $A$).

\subsubsection{Geschlecht von Neugeborenen}
Modellierung durch den Wahrscheinlichkeitsraum $(\Omega, \Sigma, P)$ mit\\
$\Omega = \{W,M\}$\\
$\Sigma = \mathbb{P}(\Omega\}$\\
$P: \Sigma \to [0,1]$ mit\\
$P(\Omega) = 1, P(\emptyset)=0$\\
$P(\{W\}) = p, P(\{M\}) = 1-p$, wobei $p \in [0,1]$.\\
Wahl von $p$: länderspezifisch auf Basis relativer Häufigkeiten, z.B. für Deutschland $p=0.4863$ (1970-1999).

\subsection{Eigenschaften von W-Maßen}
\begin{theorem}[Eigenschaften von W-Maßen]
    Sei $(\Omega, \Sigma, P)$ ein W-Raum, und seien $A, B, A_1, A_2, \dots \in \Sigma$.
    \begin{enumerate}
        \item Ist $A \subseteq B$, so gilt $P(B) = P(A) + P(B\setminus A)$ und
            $P(A) \leq P(B)$ (Monotonie)
        \item $P(A \cup B) = P(A) + P(B) - P(A\cap B)$
        \item $P\left(\cup_{i=1}^\infty A_i\right) \leq \sum_{i=1}^\infty P(A_i)$ (\ts-Subadditivität)
    \end{enumerate}
\end{theorem}

