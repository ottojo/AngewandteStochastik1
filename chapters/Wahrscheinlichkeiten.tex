\chapter{Wahrscheinlichkeiten}
\section{Wahrscheinlichkeitsräume}
Ziel dieses Abschnittes ist die Mathematische Modellierung von Zufallsexperimenten

\subsection{Ergebnisraum $\Omega$}
Der Ergebnisraum $\Omega$ ist die Menge aller möglichen Ergebnisse $\omega$ des
Zufallsexperimentes.

\paragraph{Beispiel} Eine Münze wird geworfen. $\Omega = \{K, Z\}$ ($K$opf, $Z$ahl), oder auch
$\Omega = \{0, 1\}$.

\paragraph{Beispiel} Ein Würfel wird geworfen. $\Omega = \{1,2,3,4,5,6\}$

\paragraph{Beispiel} Anzahl täglicher Bestellungen eines Artikels:
$\Omega = \{0,1,2,\dots\} = \mathbb{N}_0$

\paragraph{Beispiel} Temperatur am Schwörmontag an der Ulmer Adenauerbrücke:
$\Omega = [-50,50]$ oder $\Omega = \mathbb{R}$
\\

Beachte:
\begin{itemize}
    \item Bei der Wahl des Ergebnisraums gibt es kein \enquote{richtig} oder \enquote{falsch}.
          Das Ziel ist, einen einfachen aber adäquaten Raum zu wählen.
      \item Der Ergebnisraum kann unterschiedliche Kardinalität haben
            (in den Beispielen: endlich, abzählbar unendlich, überabzählbar unendlich).
\end{itemize}

\subsection{Ereignisse}
Motivation: Oft ist nicht das tatsächliche Ergebnis des Experiments interessant, sondern nur
ob das Ergebnis in eine vorgegebene Menge von Ergebnissen $A$ fällt.

\paragraph{Beispiel: Würfeln} Augenzahl ist gerade: $A = \{2,4,6\}$

\paragraph{Beispiel: Anzahl täglicher Bestellungen} Vorrat von 100 Artikeln wird nicht
überschritten: $A = \{0,1,2,\dots,100\}$.

\paragraph{Beispiel: Temperatur} Temperatur beträgt über $25\degree$:
$A = [25,50]$ (falls $\Omega = [-50,50])$ oder $A= [25, \infty)$ (falls $\Omega = \mathbb{R}$)
\\

Diese Teilmengen$A$ von $\Omega$, denen Wahrscheinlichkeiten zugeordnet werden sollen,
heißen \emph{Ereignisse}.
Man sagt \enquote{Das Ergebnis A tritt ein}, falls das Ergebbnis $\omega$ des Zufallexperimentes
in dieser Menge $A$ liegt (d.h. $\omega \in A$).
Mittels Mengenoperationen können Ereignisse zu neuen Ereignissen verknüpft werden.

\begin{itemize}
    \item $A \cup B$: Ereignis $A$ oder Ereignis $B$ tritt ein.

    \item $A \cap B$: Ereignis $A$ und Ereignis $B$ treten ein.

    \item $A\setminus B$: Ereignis $A$, nicht aber Ereignis $B$ tritt ein.

    \item $A^\complement = \Omega\setminus A$: Ereignis $A$ tritt nicht ein.

    \item $\bigcup\limits_{i=1}^\infty A_i$: Mindestens eines der Ereignisse
        $A_1, A_2, \dots$ tritt ein.

    \item $\bigcap\limits_{i=1}^\infty A_i$: Alle Ereignisse $A_1, A_2, \dots$
        treten ein.
\end{itemize}

\paragraph{Beispiel} Münze wird zwei Mal geworfen.\\
$\Omega = \{KK, KZ, ZK, ZZ\}$\\
Sei $A_1$ das Ereignis \enquote{Im ersten Wurf fällt Kopf} und\\
sei $A_2$ das Ereignis \enquote{Im zweiten Wurf fällt Kopf}. Dann gilt\\
$A_1 = \{KK, KZ\}$, $A_2 = \{KK, ZK\}$.\\
Die Menge $A_1 \cup A_2 = \{KK, KZ, ZK\}$ ist das Ereignis \enquote{Es fällt mindestens ein Mal
Kopf}.\\
Die Menge $A_1 \cap A_2 = \{KK\}$ ist das Ereignis \enquote{Es fällt zwei Mal Kopf}.\\

