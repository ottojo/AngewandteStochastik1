\section{Sigma-Algebra}
Welchen Teilmengen $A$ von $\Omega$ sollen Wahrscheinlichkeiten $P(A)$ zugeordnet werden?
Ideal wäre: man definiert $P(A)$ für alle $A \subset \Omega$, das heißt für alle
$A \in \mathbb{P}(\Omega) := \{B:B\subset\Omega\}$, die Potenzmenge von $\Omega$.
Das geht, falls $\Omega$ endlich oder anzählbar ist.
Es ist aber im Allgemeinen nicht möglich
(z.B. falls $\Omega = \mathbb{R}$).
Deswegen beschränkt man sich auf ein Teilsystem $\Sigma \subseteq \mathbb{P}(\Omega)$ von
Ereignissen.
\paragraph{Forderungen an $\Sigma$:}
Mengenoperationen mit Ereignissen liefern wieder Ereignisse.

\begin{definition}
    Sei $\Omega \neq \emptyset$ eine beliebige Menge.
    Eine Menge $\Sigma \subseteq \mathbb{P}(\Omega)$ heißt eine $\sigma$-Algebra, falls
    \begin{enumerate}
        \item $\Omega \in \Sigma$
        \item Falls $A \in \Sigma$, dann gilt $A^\complement \in \Sigma$
        \item Falls $A_1, A_2, \dots \in \Sigma$, dann gilt:
              $\bigcup\limits_{i=1}^\infty A_i \in \Sigma$
    \end{enumerate}
\end{definition}

Die Elemente der $\sigma$-Algebra heißen \emph{Ereignis}.
Es folgt, dass $\emptyset$ ein Element jeder $\sigma$-Algebra ist. Dieses Ereignis heißt
\emph{unmögliches Ereignis} und tritt nie ein.
Es folgt, dass \emph{endliche Vereinigungen von Ereignissen} auch Ereignisse sind.

\paragraph{Beispiel: Münzwurf}
$\Omega = \{K, Z\}$. Die Mengen\\
$\Sigma_1=\{\Omega, \emptyset\}$ und\\
$\Sigma_2=\{\Omega, \emptyset, \{K\}, \{Z\}\} = \mathbb{P}(\Omega)$ sind $\sigma$-Algebren.\\
Die $\sigma$-Algebra $\Sigma_1$ heißt die \emph{triviale $\sigma$-Algebra}.\\
Beachte: $K \notin \Sigma_2$, aber $\{K\} \in \Sigma_2$, Ergebnisse sind \emph{keine} Ereignisse!

\paragraph{Beispiel:} $\Omega = \{1,2,3\}$\\
Dann sind die Mengen\\
$\Sigma_1 = \{\Omega, \emptyset\}$,\\
$\Sigma_2 = \{\Omega, \emptyset, \{1\}, \{2,3\}\}$,\\
$\Sigma_3 = \{\Omega, \emptyset, \{2\}, \{1,3\}\}$,\\
$\Sigma_4 = \{\Omega, \emptyset, \{3\}, \{1,2\}\}$,\\
$\Sigma_5 = \{\Omega, \emptyset, \{1\}, \{2\}, \{2,3\}, \{1,3\}, \{1,2\}, \{3\} \} = \mathbb{P}(\Omega)$
$\sigma$-Algebren

\begin{theorem}
    Für jede $\sigma$-Algebra $\Sigma$ gilt
    \begin{enumerate}
        \item $\emptyset \in \Sigma$
        \item falls $A, B \in \Sigma$ dann gilt: $A \cup B, A \cap B, A \setminus B \in \Sigma$
        \item falls $A_1,A_2,\dots \in \Sigma$, dann gilt $\bigcap\limits_{i=1}^\infty A_i \in \Sigma$
    \end{enumerate}
\end{theorem}
