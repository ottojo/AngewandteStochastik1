\section{Ergebnisraum Omega}
Der Ergebnisraum $\Omega$ ist die Menge aller möglichen Ergebnisse $\omega$ des
Zufallsexperimentes.

\paragraph{Beispiel} Eine Münze wird geworfen. $\Omega = \{K, Z\}$ ($K$opf, $Z$ahl), oder auch
$\Omega = \{0, 1\}$.

\paragraph{Beispiel} Ein Würfel wird geworfen. $\Omega = \{1,2,3,4,5,6\}$

\paragraph{Beispiel} Anzahl täglicher Bestellungen eines Artikels:
$\Omega = \{0,1,2,\dots\} = \mathbb{N}_0$

\paragraph{Beispiel} Temperatur am Schwörmontag an der Ulmer Adenauerbrücke:
$\Omega = [-50,50]$ oder $\Omega = \mathbb{R}$
\\

Beachte:
\begin{itemize}
    \item Bei der Wahl des Ergebnisraums gibt es kein \enquote{richtig} oder \enquote{falsch}.
          Das Ziel ist, einen einfachen aber adäquaten Raum zu wählen.
    \item Der Ergebnisraum kann unterschiedliche Kardinalität haben
          (in den Beispielen: endlich, abzählbar unendlich, überabzählbar unendlich).
\end{itemize}
