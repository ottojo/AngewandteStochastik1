\section{Beispiele zum Wahrscheinlichkeitsraum}
\subsection{Münzwurf}
Modellierung durch den Wahrscheinlichkeitsraum $(\Omega, \Sigma, P)$ mit\\
$\Omega = \{K,Z\},$\\
$\Sigma = \mathbb{P}(\Omega) = \{ \Omega, \emptyset, \{K\}, \{Z\}\}$\\
$P: \Sigma \to [0,1]$ mit\\
$P(\Omega) = 1, P(\emptyset) = 0, P(\{K\}) = 0.5, P(\{Z\}) = 0.5$

\subsection{Würfeln}
Modellierung durch den Wahrscheinlichkeitsraum $(\Omega, \Sigma, P)$ mit\\
$\Omega = \{1,2,\dots,6\}$\\
$\Sigma =\mathbb{P}(\Omega)$\\
$P: \Sigma \to [0,1]$ mit\\
$P(\Omega) = 1, P(\emptyset) = 0$,\\
$P(\{1\})=\frac{1}{6}, P(\{2\})=\frac{1}{6}, \dots$\\
$P(\{1, 2\})=\frac{1}{3}, P(\{1,2,3\})=\frac{1}{2}$\\
Also kann man kürzer schreiben $P(A) = \frac{|A|}{|\Omega|}$, mit $|A|$ der Anzahl der Elemente in $A$
(Kardinalität von $A$).

\subsection{Geschlecht von Neugeborenen}
Modellierung durch den Wahrscheinlichkeitsraum $(\Omega, \Sigma, P)$ mit\\
$\Omega = \{W,M\}$\\
$\Sigma = \mathbb{P}(\Omega\}$\\
$P: \Sigma \to [0,1]$ mit\\
$P(\Omega) = 1, P(\emptyset)=0$\\
$P(\{W\}) = p, P(\{M\}) = 1-p$, wobei $p \in [0,1]$.\\
Wahl von $p$: länderspezifisch auf Basis relativer Häufigkeiten, z.B. für Deutschland $p=0.4863$ (1970-1999).
