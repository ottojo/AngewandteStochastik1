\section{Bedingte Wahrscheinlichkeiten}
Die bedingte Wahrscheinlichkeit von $A$ gegeben $B$ ist definiert als
\begin{equation}
    P(A|B) = \frac{P(A\cap B)}{P(B)}
\end{equation}

\begin{theorem}[Totale Wahrscheinlichkeit, Satz von Bayes]
    Sei \wraum\ ein W-Raum und seien $B_1, \dots, B_n \in \tS$ paarweise
    disjunkt mit $P(B_i) > 0$ für $i=1, \dots, n$ und
    $B_1 \cup \dots \cup B_n = \tO$
    \begin{enumerate}
        \item Satz von der totalen Wahrscheinlichkeit:\\
              Für alle $A \in \tS$ gilt:
              \begin{equation*}
                  P(A) = \sum P(A|B_i) \cdot P(B_i)
              \end{equation*}
        \item Satz von Bayes:\\
              Für alle $A \in \tS$ mit $P(A) > 0$ und alle $i=1, \dots, n$ gilt:
              \begin{equation*}
                  P(B_i|A)
                  = \frac{P(A|B_i) P(B_i)}{P(A)}
                  = \frac{P(A|B_i) P(B_i)}{\sum_j P(A|B_j) \cdot P(B_j)}
              \end{equation*}
    \end{enumerate}
\end{theorem}

\paragraph{Beachte:}
Für $A,B \in \tS$ gilt mit $P(B)>0$:
\begin{equation*}
    P(A|B) = \frac{P(A \cap B)}{P(B)}
    = \frac{P(B) - P(A^\complement \cap B)}{P(B)}
    = 1 - P(A^\complement | B)
\end{equation*}
