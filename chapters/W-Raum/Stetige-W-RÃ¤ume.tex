\section{Stetige W-Räume}
In diesem Abschnitt wählen wir $\tO = \mathbb{.R}$
In Fällen, in denen es intuitiv wäre, als Ergebnisraum ein Intervall zu wählen,
werden wir trotzdem $\tO = \mathbb{.R}$ wählen, und das Intervall durch passendes
W-Maß modellieren.

Wie sehen hier nun \tO\ und $P$ aus?

\subsection{Auswahl von Sigma}
$\tS = P(\tO)$ ist nicht mehr möglich. $P(\mathbb{R})$ enthält viele pathologische Ereignisse,
also ist $\tS = P(\mathbb{R})$ zu groß! Statdessen wird eine kleinere \ts-Algebra gewählt.

\begin{definition}[Borel-\ts-Algebra]
    Die kleinste \ts-Algebra auf $\mathbb{R}$, die alle Intervalle $[a,b]$ mit $a<b$
    enthält, heißt die Borel-\ts-Algebra und wird mit $B(\mathbb{R})$ bezeichnet.
\end{definition}

"Kleinste" bedeutet: Für alle \ts-Algebren \tS\ auf \real, die alle Intervalle $[a,b]$ mit
$a<b$ enthält, gilt: $B(\real) \subseteq  \tS$.

\paragraph{Bemerkungen}
\begin{itemize}
    \item $B(\real)$ enthält alle "relevanten" Teilmengen von \real, sowie die Mengen $[a,b]$,
          $[a,b)$, $(-\infty, b]$, $\{a\}$ usw. für $a \leq b$.
    \item $B(\real) \neq P(\real)$
\end{itemize}

\subsection{Definition von P}
\paragraph{Idee}
\begin{itemize}
    \item Aus der Folge $p_1, p_2, \dots$ wird eine Funktion $p: \real \to \real$
    \item Aus der Summe ($P(A)=\sum_{\w_i \in A} p_i$) wird ein Integral
\end{itemize}

\begin{definition}[Wahrscheinlichkeitsdichte]
    Eine Funktion $p: \real \to \real$ heißt Wahrscheinlichkeitsdichte (Dichte), falls
    \begin{enumerate}
        \item $p(x) \geq 0 \;\; \forall x\in \real$
        \item $\displaystyle\int_{-\infty}^\infty p(x)\; dx = 1$
    \end{enumerate}
\end{definition}

\paragraph{Idee:}
Definiere $P$ so, dass für alle $[a,b]$ mit $a \leq b$ gilt:
\begin{equation*}
    P([a,b]) = \int_a^b p(x) dx
\end{equation*}

\begin{theorem}
    \label{theorem:wmass-aus-dichte}
    Sei $p: \real \to \real$ eine Dichte.
    Es existiert genau ein W-Maß $P$ auf $B(\real)$ mit $P([a,b]) = \displaystyle\int_a^b p(x) dx$
    für alle $a \leq b$.
\end{theorem}

\begin{definition}
    Sei $p: \real \to \real$ eine Dichte und sei $P$ ein W-Maß auf $B(\real)$ mit
    $P([a,b]) = \displaystyle\int_a^b p(x) dx$ für alle $a \leq b$. Dann heißt P
    \emph{absolut stetig} mit Dichte $p$. Man sagt: $P$ besitzt Dichte $p$.
\end{definition}

\begin{theorem}
    \label{theorem:dichte}
    Sei $P$ absolut stetig mit Dichte $p$. Dann gilt:
    \begin{enumerate}
        \item $P(\{a\}) = 0 \; \forall a \in \real$
        \item $P([a,b]) = P((a,b)) = P((a,b]) = P([a,b)) = \displaystyle\int_a^b p(x) dx$
    \end{enumerate}
\end{theorem}

\paragraph{Beachte:}
\autoref{theorem:dichte} liefgert, dass $P$ nicht eindeutig durch die Werte $P(\{a\})$,
$a \in \real$ definiert ist.
Es folgt aus \autoref{theorem:wmass-aus-dichte}, dass $P$ duch die Dichte $p$
eindeutig bestimmt ist.\\

Es folgen einige wichtige absolut stetige W-Maße.

\subsection{Stetige Gleichverteilung}
Seien $a < b$ und
\begin{equation*}
    p(x) = \begin{cases}
        \frac{1}{b-a} & , x \in [a,b] \\
        0             & , sonst
    \end{cases}
\end{equation*}

\begin{figure}[H]
    \centering
    \begin{tikzpicture}
        \begin{axis}[
                axis lines=middle,
                xmin=0, xmax=1, % x scale
                ymin=-0.1, ymax=1, % y scale
                xtick={0.2, 0.8},
                xticklabels={a, b},
                ytick={0, 0.8},
                yticklabels={$0$, $\displaystyle\frac{1}{b-a}$},
                domain=0:1
            ]
            \addplot [blue,no marks, thick, domain=0:0.2] {0};
            \addplot [blue,no marks, thick, domain=0.2:0.8] {0.8};
            \addplot [blue,no marks, thick, domain=0.8:1] {0};
        \end{axis}
    \end{tikzpicture}
    \caption{Stetige Gleichverteilung, Wahrscheinlichkeitsdichte}
    \label{}
\end{figure}

Besitzt $P$ Dichte $p$, so nennt man $P$ \emph{Stetige Gleichverteilung} auf $[a,b]$.
Man schreibt $P=U([a,b])$.

\subsection{Exponentialverteilung}
Sei $\lambda > 0$ und
\begin{equation*}
    p(x) = \begin{cases}
        0                      & , falls x < 0    \\
        \lambda e^{-\lambda x} & , falls x \geq 0
    \end{cases}
\end{equation*}
Dann ist $p$ eine Dichte, da
\begin{enumerate}
    \item $p(x) > 0 \;\; \forall x \in \real$
    \item $\int p(x) dx = \lambda \int e^{-\lambda x} dx = \lambda\left[-e^{-\lambda x} \cdot \frac{1}{\lambda}\right]_0^\infty = 1$
\end{enumerate}
Besitzt $P$ die Dichte $p$, so nennt man $P$ Exponentialverteilung mit Parameter $\lambda>0$.
Man schreibt $P=\text{Exp}(\lambda)$.

\paragraph{Bemerkung}
\begin{itemize}
    \item Eigenschaft der "Gedächtnislosigkeit" ($\to$ später)
    \item Geeignet zur Modellierung von Lebensdauern und Wartezeiten
\end{itemize}

\begin{figure}[H]
    \centering
    \begin{tikzpicture}
        \begin{axis}[
                axis lines=middle,
                samples=100,
                xmin=-1, xmax=5,
                ymin=-0.1, ymax=1.3,
                xtick={0},
                xticklabels={$0$},
                ytick={0, 1},
                yticklabels={$0$, $\lambda$},
                domain=0:5,
                smooth
            ]
            \addplot [blue, no marks, thick] {e ^ -x};
        \end{axis}
    \end{tikzpicture}
    \caption{Exponentialverteilung, Wahrscheinlichkeitsdichte}
\end{figure}

\subsection{Normalverteilung}
Für $\mu \in \real$ und $\ts^2 > 0$ setze
\begin{equation*}
    p(x) = \frac{1}{\sqrt{2 \pi \ts^2}} \exp\left( \frac{-(x-\mu)^2}{2 \ts^2} \right) \;\; \forall x\in\real
\end{equation*}

\paragraph{Beachte}
Auch hier gilt $\int p(x) dx = 1$ und $p(x) > 0$, $p$ ist also eine Dichte.
\\
\\
\noindent
Besitzt $P$ Dichte $p$, so nennt man $P$ die Normalverteilung mit Parametern
$\mu$ und $\ts^2$. $P=N(\mu, \ts^2)$.

Falls $\mu = 0$ und $\ts^2 = 1$ gilt $p(x) = \frac{1}{\sqrt{2 \pi}} \exp\left(\frac{-x^2}{2}\right) \forall x\in\real$.
Man nennt $P=N(0,1)$ die \emph{Standardnormalverteilung}.

\paragraph{Beachte}
\begin{enumerate}
    \item $N(0,1)$ spielt eine zentrale Rolle in der Stochastik, siehe der zentrale Grenzwertsatz ($\to$ später)
    \item Man verwendet die Normalverteilung oft zur Modellierung von Messfehlern, Blattlängen,
          Blutdruckwerten, Temperaturen, etc.
\end{enumerate}

\begin{figure}[H]
    \centering
    \begin{tikzpicture}
        \begin{axis}[
                axis lines=middle,
                samples=200,
                xmin=-1, xmax=10,
                ymin=-0.1, ymax=0.7,
                xtick={0,5},
                xticklabels={$0$, $\mu$},
                ytick={0, 0.199, 0.399},
                yticklabels={$0$,, $\displaystyle\frac{1}{\sqrt{2 \pi \ts^2}}$},
                domain=-1:10,
                smooth,
                thick
            ]

            \addplot [blue] {gauss(5,1)};
            \addlegendentry{$N(\mu,\sigma^2=1)$}

            \addplot [red] {gauss(5,2)};
            \addlegendentry{$N(\mu,\sigma^2=2)$}
        \end{axis}
    \end{tikzpicture}
    \caption{Normalverteilung, Wahrscheinlichkeitsdichte}
    \label{}
\end{figure}
