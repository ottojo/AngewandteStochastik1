\section{Unabhängigkeit}
Sei \wraum\ ein W-Raum und $A,B \in \tS$ mit $P(A), P(B) > 0$.

\paragraph{Idee:}
Unabhängigkeit von $A$ und $B$ soll bedeuten, dass das Eintreten von $A$ (bzw $B$)
keinen Einfluss auf $P(A)$ (bzw $P(B)$) hat, d.h.
\begin{equation*}
    P(A|B) \defeq \frac{P(A \cap B)}{P(B)} \shouldeq P(A)
\end{equation*}
\begin{equation*}
    P(B|A) \defeq \frac{P(B \cap A)}{P(A)} \shouldeq P(B)
\end{equation*}
also muss
\begin{equation*}
    P(A \cap B) = P(A) \cdot P(B)
\end{equation*}
gelten.

\begin{definition}[Unabhängigkeit]
    Sei \wraum\ ein W-Raum und $A, B \in \tS$.
    Dann heißen A und B \emph{unabhängig}, falls gilt:
    \begin{equation}
        P(A \cap B) = P(A) \cdot P(B)
    \end{equation}
\end{definition}

\begin{definition}[Unabhängigkeit mehrerer Ereignisse]
    Sei \wraum\ ein W-Raum und $A_1, A_2, \dots, A_n \in \tS$.
    Dann heißen $A_1, A_2, \dots, A_n$
    \begin{enumerate}
        \item paarweise unabhängig, falls $A_i$ und $A_j$ unabhängig sind
              für alle $i \neq j$.
        \item unabhängig, falls für alle $T \subseteq \{1, \dots, n\}$
              mit $|T| \geq 2$ gilt:
              \begin{equation*}
                  P\left(\bigcap_{i \in T} A_i \right) = \prod_{i \in T} P\left( A_i \right)
              \end{equation*}
    \end{enumerate}
\end{definition}

\paragraph{Bemerkung:}
Unabhängigkeit impliziert paarweise Unabhängigkeit. Umgekehrt nicht!
