\section{Diskrete Wahrscheinlichkeitsräume}
Bisher war \tO\ endlich. In diesem Abschnitt soll \tO\ abzählbar unendlich sein, d.h.
$\tO = \{ \w_1, \w_2, \dots\}$, $\w_i \neq \w_j$ für $i \neq j$.

\subsection{Poisson Verteilung}
$\tO = \mathbb{N}_0$, $\tS = P(\tO)$
\begin{align*}
    P(\{k\}) & = \frac{\lambda^k}{k!} \cdot e^{-\lambda}, \; k \in \mathbb{N}_0, \; \lambda > 0 \\
    P(A)     & = \sum_{k \in A} P(\{k\}), \; A \in \tS
\end{align*}
Mit der Reihenentwicklung der Exponentialfunktion lässt sich zeigen dass $P(\tO) = 1$.
$P$ heißt Poissonverteilung mit Parameter $\lambda > 0$: $P = P(\lambda)$

\begin{theorem}[Poisson Approximation]
    Sei $p_1, p_2, \dots$ eine Folge mit $p_j \in [0, 1]$ für $j \in \mathbb{N}$ und
    $n \cdot p_n \xrightarrow[n \to \infty]{} \lambda > 0$.
    Dann gilt
    \begin{equation*}
        B(n, p_n)(\{k\}) \xrightarrow[n \to \infty]{} P(\lambda)(\{k\})
    \end{equation*}
    für alle $k \in \mathbb{N}_0$. D.h.
    \begin{equation*}
        \binom{n}{k} \cdot p_n^k \cdot (1-p_n)^{n-k} \xrightarrow[n \to \infty]{} \frac{\lambda^k}{k!} \cdot e^{-\lambda}
    \end{equation*}
    Das heißt die Poissonverteilung taucht auf, wenn viele Versuche mit kleiner Erfolgswahrscheinlichkeit
    durchgeführt werden, d.h. falls $n$ groß und $p$ klein ist.
\end{theorem}

\subsection{Geometrische Verteilung}
Hier wird ein Zufallsexperiment so oft durchgeführt, bis das erste mal das Ergebnis "Erfolg" auftritt.
\begin{align*}
    \tO      & = \mathbb{N}                               \\
    \tS      & = P(\tO)                                   \\
    P(\{k\}) & = (1-p)^{k-1} \cdot p, \; k \in \mathbb{N} \\
    P(A)     & = \sum_{k \in A} P(\{k\}), \; A \in \tS
\end{align*}
Man schreibt $P = G(p)$ für eine geometrische Verteilung mit Erfolgswahrscheinlichkeit $p \in (0,1]$.
