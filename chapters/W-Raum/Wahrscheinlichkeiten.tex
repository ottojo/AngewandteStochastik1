\section{Wahrscheinlichkeiten}

Wir definieren $P(A)$ für alle $A \in \Sigma$.
\begin{definition}[Disjunkte Mengen]
    Mengen $A, B$ heißen disjunkt, falls $A\cap B = \emptyset$\\
    Mengen $A_1, A_2, \dots$ heißen paarweise disjunkt (p.d.) falls für alle $i \neq j$ gilt
    $A_i \cap A_j = \emptyset$
\end{definition}

Idee: P soll folgende Eigenschaften haben:
\begin{enumerate}
    \item Für alle $A\in \Sigma$ gilt: $0 \leq P(A)\leq 1$
    \item $P(\Omega) = 1$ und $P(\emptyset) = 0$
    \item $P(A^\complement) = 1-P(A)$ für alle $A\in \Sigma$
    \item falls $A,B\in\Sigma$ disjunkt sind, so gilt
          $P(A\cup B) = P(A) + P(B)$.
    \item falls $A_1, A_2,\dots \in \Sigma$ p.d., so gilt
          $P(\bigcup\limits_{i=1}^\infty A_i) = \sum_{i=1}^\infty P(A_i)$
\end{enumerate}

Das motiviert folgende Definition

\begin{definition}[Wahrscheinlichkeitsmaß]
    Sei $\Omega \neq \emptyset$ und \tS\ eine \ts-Algebra auf \tO.
    Eine Abbildung $P: \Sigma \to [0,1]$ heißt Wahrscheinlichkeitsmaß (W-Maß) auf \tS\ falls
    \begin{enumerate}
        \item $P(\Omega) = 1$
        \item Sind $A_1, A_2,\dots \in \Sigma$ paarweise disjunkt, dann gilt
              $P\left(\bigcup_{i=1}^\infty A_i\right) = \sum_{i=1}^\infty P(A_i)$
              (\ts-Additivität)
    \end{enumerate}
\end{definition}
\noindent
Das Tripel $\left( \Omega, \Sigma, P\right)$ heißt \emph{Wahrscheinlichkeitsraum} (W-Raum).
