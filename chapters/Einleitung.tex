\chapter{Einleitung}

Zentrales Objekt der Stochastik ist das \emph{Zufallsexperiment}:
Dies ist ein Experiment, bei dem mehrere Ergebnisse eintreten können. Es ist nicht
vorhersagbar, welches Ergebnis eintritt.

Beispiele sind Münzwurf, oder das Werfen eines Würfels. Die Ergebnisse sind hier Kopf/Zahl
respektive die Zahlen 1 bis 6.
Ein Zufallsexperiment hat keine \emph{deterministische Regelmäßigkeit}:
Wiederholungen eines Zufallsexperiments können verschiedene Ergebnisse haben.
Ein Zufallsexperiment hat aber eine \emph{statistische Regelmäßigkeit}:
Relative Häufigkeiten der einzelnen Ergebnisse stabilisieren sich nach vielen Wiederholungen.
Beim 1000-facher Wiederholung des Münzwurfs zum Beispiel ist die \emph{relative Häufigkeit}
\begin{equation*}
    = \frac{\text{Anzahl von Kopf}}{1000} \approx \frac{500}{1000} = \frac{1}{2}
\end{equation*}

Die statistische Regelmäßigkeit ermöglicht es, Vorhersagen bestimmter Art zu machen:
Beim 1000-fachen Münzwurf wird Kopf mindestens 450 Mal auftreten mit Wahrscheinlichkeit $99.8\%$.
